\RequirePackage{fixltx2e}
\documentclass[pagesize=auto]{scrartcl}

\usepackage{blindtext}
\usepackage{etex}
\usepackage{lmodern}
\usepackage[T1]{fontenc}
\usepackage{textcomp}
\usepackage{array}
\usepackage{xcolor}
\usepackage{tikz}
\usepackage{microtype}
\usepackage{hyperref}
\usepackage{copyrightbox}

\newcommand*{\mail}[1]{\href{mailto:#1}{\texttt{#1}}}
\newcommand*{\pkg}[1]{\textsf{#1}}
\newcommand*{\cls}[1]{\textsf{#1}}
\newcommand*{\cs}[1]{\texttt{\textbackslash#1}}
\makeatletter
\newcommand*{\cmd}[1]{\cs{\expandafter\@gobble\string#1}}
\makeatother
\newcommand*{\env}[1]{\texttt{#1}}
\newcommand*{\opt}[1]{\texttt{#1}}
\newcommand*{\meta}[1]{\textlangle\textsl{#1}\textrangle}
\newcommand*{\marg}[1]{\texttt{\{}\meta{#1}\texttt{\}}}
\newcommand*{\oarg}[1]{\texttt{[}\meta{#1}\texttt{]}}

\addtokomafont{title}{\rmfamily}

\title{The \pkg{copyrightbox} package\thanks{This manual corresponds to
\pkg{copyrightbox}~v0.1b, dated~Mar 22, 2024.}}
\author{Thomas Fischer\thanks{\mail{thomas.fischer@t-fischer.net}}\\%
Ives van der Flaas\thanks{\mail{ives.vdf@gmail.com}}\\%
Johannes B{\"o}ttcher\thanks{\mail{johannesbottcher@gmail.com}}\\%
Mark A. Greenwood\thanks{\mail{m.a.greenwood@sheffield.ac.uk}}%
}
\date{\today}


\begin{document}

\maketitle

\noindent
\pkg{copyrightbox.sty} provides the command \cmd{\copyrightbox}, which is used
to put a small amount of text, usually a copyright notice, close to an image.
This text can be either below the image, it can be to the right of the image
(and rotated counterclockwise respective to normal text orientation) and it
can be to the left of the image, also rotated counterclockwise.


The command  \cmd{\copyrightbox} has two obligatory parameters, the first
being a box (usually an image loaded with the \cmd{\includegraphics} command
from the \pkg{graphicx} package) and the second being the copyright message.
It also has an optional parameter that determines the position of the
copyright notice relative to the image:
\begin{quote}
	\verb=\copyrightbox[=\meta{placement}\verb=]{=\meta{image}\verb=}{=\meta{text}\verb=}=
\end{quote}

Allowed \meta{placement}s are
\begin{description}
	\item[\texttt{l}]The copyright notice is placed to the left of the image.
	\item[\texttt{r}]The copyright notice is placed to the right of the image.
	                 This is also the default if this parameter is omitted, i.e.\ it is equivalent to
	                 \begin{quote}
	                 	\verb=\copyrightbox{=\meta{image}\verb=}{=\meta{text}\verb=}=
	                 \end{quote}
	\item[\texttt{b}]The copyright notice is placed below the image.
\end{description}

Note that a \cmd{\copyrightbox} will be bigger than the image it encompasses, since it
will also fit the copyright text.

\clearpage
\section{Examples}
\setcounter{secnumdepth}{1}
\subsection{Example 1}
The command

\begin{verbatim}
 \copyrightbox
      {\includegraphics[scale=0.5]{coin}}
      {Image put in the public domain by the U.S. Mint.}
\end{verbatim}
produces:
\begin{center}
	\copyrightbox
		{\includegraphics[scale=0.5]{coin}}
		{Image put in the public domain by the U.S. Mint.}
\end{center}

\subsection{Example 2}
The command

\begin{verbatim}
 \copyrightbox[b]
      {\includegraphics[scale=0.5]{coin}}
      {Image put in the public domain by the U.S. Mint.}
\end{verbatim}
produces:
\begin{center}
	\copyrightbox[b]
		{\includegraphics[scale=0.5]{coin}}
		{Image put in the public domain by the U.S. Mint.}
\end{center}


\subsection{Example 3}
The command

\begin{verbatim}
 \copyrightbox[l]
      {\includegraphics[scale=0.5]{coin}}
      {Image put in the public domain by the U.S. Mint.}
\end{verbatim}
produces:
\begin{center}
	\copyrightbox[l]
		{\includegraphics[scale=0.5]{coin}}
		{Image put in the public domain by the U.S. Mint.}
\end{center}


\subsection{Example 4}
\cmd{\copyrightbox} does not require \cmd{\includegraphics} images, you can
just as easily generate your own images with something like \pkg{TikZ} or the
\LaTeX\ \env{picture} environment.

For example, the command

\begin{verbatim}
\copyrightbox[r] { \tikz \filldraw[fill=green] (0,0) circle (2); }
{Image created by the author.}
\end{verbatim}
produces:
\begin{center}
\copyrightbox[r] { \tikz \filldraw[fill=green] (0,0) circle (2); }
{Image created by the author.}
\end{center}

\setcounter{secnumdepth}{2}
\section{Notes}

\subsection{Changing the Font}
It's possible to change every aspect of the font used in the copyright message
using \cmd{\setcopyrightfont}, for example: 

\begin{verbatim}
\setcopyrightfont{\tiny\itshape}
\end{verbatim}
This can be placed anywhere in your \texttt{.tex} file and will
typeset copyright notes in tiny italics:
\begin{center}
	\setcopyrightfont{\tiny\itshape}
\copyrightbox[r] { \tikz \filldraw[fill=gray] (0,0) rectangle (1,2); }
{Image created by the author.}
\end{center}

It is also possible to redefine the internal commands, which is
not recommended. Still, here a short example:
\begin{verbatim}
\makeatletter
\renewcommand{\CRB@setcopyrightfont}{%
	\footnotesize 
	\color{red!33}
	\scshape
}
\makeatother

\copyrightbox[r] { \tikz \filldraw[fill=green] (0,0) circle (2); }
{Image created by the author.}
\end{verbatim}
This will generate the copyright notice in small caps in a reddish tint:
\begin{center}
\makeatletter
\renewcommand{\CRB@setcopyrightfont}{%
	\footnotesize 
	\color{red!33}
	\scshape
}
\makeatother

\copyrightbox[r] { \tikz \filldraw[fill=green] (0,0) circle (2); }
{Image created by the author.}
\end{center}

Of course, it's also possible to select a font using the New Font Selection
System in \LaTeXe, for example, to select Helvetica\footnote{This font might not be available on your system.} size
4:

\begin{verbatim}
\setcopyrightfont{%
\usefont{T1}{phv}{m}{n}\fontsize{4}{5}\selectfont
}

\copyrightbox[r] { \tikz \filldraw[fill=green] (0,0) circle (2); }
{Image created by the author.}
\end{verbatim}

\begin{center}
\setcopyrightfont{%
\usefont{T1}{phv}{m}{n}\fontsize{4}{5}\selectfont
}

\copyrightbox[r] { \tikz \filldraw[fill=green] (0,0) circle (2); }
{Image created by the author.\blindtext}
\end{center}

\subsection{Justification}
Although it's not exactly recommended, it is possible to place relatively
large amounts of text near images. Long amounts of text are usually nicer to
look at when justified, and to do so \cmd{\setcopyrightparstyle}
to something other than \cmd{\raggedright}.

Place

\begin{verbatim}
\setcopyrightparstyle{%
% Return to justifying text
\setlength{\rightskip}{0pt}
\setlength{\leftskip}{0pt}
}

\begin{center}
	\copyrightbox[r]
		{\includegraphics[scale=0.5]{coin}}
      {This here image was put in the public domain some time ago 
      by the U.S. Mint.}
\end{center}
\end{verbatim}

anywhere in your .tex file to produce:
\setcopyrightparstyle{%
% Return to justifying text
\setlength{\rightskip}{0pt}
\setlength{\leftskip}{0pt}
}

\begin{center}
	\copyrightbox[r]
		{\includegraphics[scale=0.5]{coin}}
      {This here image was put in the public domain some time ago by the U.S. Mint.}
\end{center}

instead of 

\setcopyrightparstyle{%
\raggedright
}

\begin{center}
	\copyrightbox[r]
		{\includegraphics[scale=0.5]{coin}}
      {This here image was put in the public domain some time ago by the U.S. Mint.}
\end{center}

Redefining the internal commands will work as well, but is not
recommended:
\begin{verbatim}
\makeatletter
\renewcommand{\CRB@setcopyrightparagraphstyle}{\raggedleft}
\makeatother
\begin{center}
\copyrightbox[r] { \tikz \filldraw[fill=gray] (0,0) rectangle (1,2); }
{There really isn't much to see here, i am sorry}
\end{center}
\end{verbatim}
\makeatletter
\renewcommand{\CRB@setcopyrightparagraphstyle}{\raggedleft}
\makeatother
\begin{center}
\copyrightbox[r] { \tikz \filldraw[fill=gray] (0,0) rectangle (1,2); }
{There really isn't much to see here, i am sorry}
\end{center}

\section{Bugs and Suggestions}
Bugreports, suggestions and patches are welcome and can be mailed to
\mail{ives.vdf@gmail.com}, or you can fork
\url{https://github.com/Ivesvdf/copyrightbox} and send me a pull request
through Github. 
\end{document}
